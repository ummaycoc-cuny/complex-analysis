\documentclass{article}
\usepackage{amsfonts}
\usepackage{amsmath}
\usepackage{amssymb}
\usepackage{fancyheadings}
\usepackage{graphicx}
\usepackage[top=1.25in, bottom=1.25in, left=1.0in, right=1.0in]{geometry}
\usepackage{hyperref}

\pagestyle{fancy}

\lhead{Complex Analysis Notes}
\rhead{Graduate Center, CUNY, 2017-2018}

\hypersetup{colorlinks=true,linkcolor=blue,urlcolor=blue} 
\setcounter{tocdepth}{3}
\title{Complex Analysis Notes\\
Graduate Center, CUNY\\
2017-2018 AY}
\date{}

\newcommand{\Mid}{\ \middle\vert\ }

\DeclareMathOperator{\interior}{int}
\DeclareMathOperator{\exterior}{ext}
\DeclareMathOperator{\len}{length}
\DeclareMathOperator{\order}{ord}
\DeclareMathOperator{\degree}{deg}
\DeclareMathOperator{\diam}{diam}

\newenvironment{topic}[1]{%
{\subsection{#1}}%
\begin{itemize}%
}{%
\end{itemize}%
}

\newcommand{\theorem}[1]{\item {\bf #1.}}
\newcommand{\corollary}[1]{\item {\bf #1.}}
\newcommand{\lemma}[1]{\item {\bf #1.}}
\newcommand{\proposition}[1]{\item {\bf #1.}}
\newcommand{\term}[1]{{\bf #1}}
\newcommand{\holo}[1]{\mathcal{O}(#1)}
\newcommand{\mero}[1]{\mathcal{M}(#1)}
\newcommand{\remark}{\item}
\newcommand{\entire}{\holo{\mathbb{C}}}
\newcommand{\udisk}{\mathbb{D}}
\newcommand{\pdisk}[2]{\mathbb{D}^\ast(#1, #2)}
\newcommand{\disk}[2]{\mathbb{D}(#1, #2)}
\newcommand{\cdisk}[2]{\overline{\mathbb{D}}(#1, #2)}
\newcommand{\sphere}[2]{\mathbb{T}{(#1, #2)}}
\newcommand{\curveproduct}{\boldsymbol{\cdot}}

\begin{document}

\maketitle

{\large These notes are drawn from the upcoming text by \href{http://qcpages.qc.cuny.edu/~zakeri/}{\underline{Saeed Zakeri}} of Queens College in NY. Some remarks, theorems, etc are paraphrased; some remarks, theorems, etc are verbatim. Notes created with the permission of the professor, who retains all copyright rights regarding his work. This is, explicitly, a derivative work made with permission of the author.
\\

\noindent Section and subsection titles exactly follow the chapters and sections of the text as of the time of the writing.}

\newpage

\tableofcontents

\newpage

\section{Rudiments of Complex Analysis}

\begin{topic}{What is a holomorphic function?}

\remark $f$ defined near $p \in \mathbb{C}$ and complex valued, is \term{complex differentiable} at $p$ if $$f^\prime(p) = \lim_{z \to p} \dfrac{f(z) - f(p)}{z - p}$$ exists. $f^\prime(p)$ is called the \term{complex derivative} of $f$ at $p$.

\remark Differentiability implies continuity.

\theorem{Differentiation Rules} $f$ and $g$ differentiable at $p$:
\begin{itemize}
\item $(f + g)^\prime(p) = f^\prime(p) + g^\prime(p)$
\item $(fg)^\prime(p) = f^\prime(p)g(p) + f(p)g^\prime(p)$,
\item If $g(p) \neq 0$ then $$\left(\frac{f}{g}\right)^\prime(p) = \dfrac{f^\prime(p)g(p) - f(p)g^\prime(p)}{(g(p))^2}$$
\end{itemize}
If $g^\prime(p)$ and $f^\prime(g(p))$ exist, then $(f \circ g)^\prime(p) = f^\prime(g(p)) g^\prime(p)$.

\remark $z = x + iy \mapsto (x, y)$ gives the canonical isomorphism $\mathbb{C} \to \mathbb{R}^2$.

\remark $f = u + iv$ can be identified with $f(x, y) = (u(x, y), v(x, y))$ -- i.e. $f : \mathbb{C} \to \mathbb{C}$ vs $f : \mathbb{R}^2 \to \mathbb{R}^2$.
$$f_x = u_x + i v_x, \hskip 0.3cm f_y = u_y + i v_y$$

\remark {\em New} partial operators: 
$$f_z = \frac{1}{2}\left(f_x - i f_y\right), \hskip 0.3cm f_{\overline{z}} = \frac{1}{2}\left(f_x + i f_y\right)$$
$$f_x = f_z + f_{\overline{z}}, \hskip 0.3cm f_y = i(f_z - f_{\overline{z}})$$

\theorem{Conditions for Differentiability} For $f$ defined near $p \in \mathbb{C}$, the following are equivalent:
\begin{itemize}
\item Complex derivative $f^\prime(p)$ exists.
\item Real derivative $Df(p)$ exists and $f_{\overline{z}} = 0$.
\item Real derivative $Df(p)$ exists with $u_x = v_y$ and $u_y = -v_x$ at $p$.
\end{itemize}
In such cases $f^\prime(p) = f_z(p) = f_x(p) = -i f_y(p)$.

\remark An angle preserving linear transformation is called \term{conformal} (shapes preserved, not scales).

\corollary{Conformality of Derivative} Suppose $f^\prime(p) \neq 0$ exists. Then $Df(p) : \mathbb{R}^2 \to \mathbb{R}^2$ is an orientation preserving conformal linear transformation.

\remark For $U \subseteq \mathbb{C}$ non-empty and open, $f : U \to \mathbb{C}$ is \term{holomorphic} in $U$, written $f \in \holo{U}$, if $f^\prime(p)$ exists $\forall p \in U$. Elements of $\entire{}$ are called \term{entire}. $\holo{U}$ is closed under sums, products, and composition. Pointwise addition and multiplication makes $\holo{U}$ a commutative ring with identity.

\theorem{Conditions for Being Holomorphic} Suppose $f = u + iv$, defined on $U$, is real differentiable. Then $f \in \holo{U}$ exactly when the following equivalent conditions hold throughout $U$:
\begin{enumerate}
\item $u_x = v_y$ and $u_y = -v_x$, the \term{Cauchy-Riemann Equations};
\item $f_{\overline{z}} = 0$, the \term{Complex Cauchy-Riemann Equation}.
\end{enumerate}
In this case, $f^\prime = f_z = f_x = -if_y$.

\end{topic}


\begin{topic}{Complex analytic functions}

\remark $\sum_{n=0}^\infty a_n (z - p)^n$, where $p, a_0, a_1, \ldots \in \mathbb{C}$, is a \term{power series}. Each power series has a \term{disk of convergence} about $p$: the series converges inside the disk, diverges outside it, and can do anything on the boundary. The disk's radius, possibly $0$ or $+\infty$, is the \term{radius of convergence} of the series.

\theorem{Radius of Convergence} The radius of convergence for $\sum_{n=0}^\infty a_n (z - p)^n$ is $$R = \dfrac{1}{\limsup_{n \to \infty} \sqrt[n]{|a_n|}} \in [0, +\infty].$$ The series converges absolutely and uniformly inside $\disk{p}{r}$, $r < R$, and diverges for $z$ with $|z - p| > R$.

\remark For $U \subseteq \mathbb{C}$ non-empty and open, $f : U \to \mathbb{C}$ is \term{complex analytic} if, whenever $\disk{p}{r} \subseteq U$, there is a power series $\sum_{n=0}^\infty a_n (z - p)^n$ converging to $f(z)$ whenever $z \in \disk{p}{r}$.

\theorem{Analytic Implies Holomorphic} A complex analytic function has derivatives of all orders, each of which is complex analytic and can be found via term-by-term differentiation. The coefficients of a complex analytic function's power series are independent of the radius of disk under consideration: $a_n = f^{(n)}(p) / n!$, $n \geq 0$ ($p$ the center of the disk under consideration).

\end{topic}


\begin{topic}{Complex integration}

\remark For $U \subseteq \mathbb{C}$ non-empty and open, $\gamma : [a, b] \to U$, continuous, is a \term{curve} in $U$ with \term{initial point} $\gamma(a)$ and \term{end point} $\gamma(b)$, and it is \term{closed} if $\gamma(a) = \gamma(b)$. $|\gamma| = \{ \gamma(t) : t \in [a, b] \} \subseteq \mathbb{C}$ is the \term{image}. $\gamma$ is \term{piecewise $C^1$} if there is $a = t_0 < t_1 < \cdots < t_n = b$ with $\gamma$ continuously differentiable on $(t_{k-1}, t_k)$ and the limits $\lim_{t \to t_{k-1}^+} \gamma^\prime(t)$ and $\lim_{t \to t_k^-} \gamma^\prime(t)$ exist.

If $\gamma$ is a curve then $\gamma^-$ is the \term{reverse} of $\gamma$, and if $\eta$ is another curve starting where $\gamma$ ends, then $\gamma \curveproduct \eta$ is the \term{product} of $\gamma$ and $\eta$ and is formed by traversing $\gamma$ and then $\eta$.

\remark Throughout the section, it will be assumed that all curves are piecewise $C^1$.

\remark For $\gamma : [a, b] \to \mathbb{C}$ a curve, $f : |\gamma| \to \mathbb{C}$ continuous, the \term{integral of $f$ along $\gamma$} is $$\int_\gamma f(z)\,dz = \int_a^b f(\gamma(t)) \gamma^\prime(t)\,dt$$ where $\gamma^\prime = d\gamma/dt$ is defined at all but finitely any points in $[a, b]$, and is invariant under orientation preserving reparameterizations of $\gamma$. Note that $$\int_\gamma f(z)\,dz = - \int_{\gamma^-} f(z)\,dz, \hskip 0.5cm \int_{\gamma \curveproduct \eta} f(z)\,dz = \int_\gamma f(z)\,dz + \int_\eta f(z)\,dz.$$ For $f = u + iv$ and $\gamma(t) = x(t) + iy(t)$, $$\int_\gamma f(z)\,dz = \int_\gamma (u\,dx - v\,dy) + i \int_\gamma (v\,dx + u\,dy).$$

\remark The \term{length} of a curve $\gamma$ is $\int_\gamma |dz| = \int_0^1 |\gamma^\prime(t)|\,dt$, denoted $\len(\gamma)$.

\remark The \term{ML-inequality} is: $$\left|\int_\gamma f(z)\,dz\right| \leq \sup_{z \in |\gamma|} |f(z)| \cdot \len(\gamma).$$

\theorem{Continuous Dependence on Vertices} Let $f : U \to \mathbb{C}$ be continuous. For $T$ a closed triangle lying in $U$, $\int_{\partial T} f(z)\,dz$ depends continuously on the vertices of $T$.

\remark $F$ is a \term{primative} of the continuous function $f : U \to \mathbb{C}$ if $F^\prime(z) = f(z)$ in $U$.

\remark $\int_{\gamma} f(z)\,dz = F(\gamma(1)) - F(\gamma(0))$.

\theorem{Existence of Primitives} $f : U \to \mathbb{C}$, continuous, has a primitive in $U$ iff $\int_\gamma f(z)\, dz = 0$ for every closed curve in $U$.

\end{topic}


\begin{topic}{Cauchy's theory in a disk}

\theorem{Primitives in Disks} Let $D \subseteq \mathbb{C}$ be an open disk and $f : D \to \mathbb{C}$ continuous. If $\int_{\partial T} f(z)\,dz = 0$ for every closed triangle $T \subseteq D$ then $f$ has a primitive in $D$.

\theorem{Goursat's Theorem} If $f \in \holo{U}$ then $\int_{\partial T} f(z)\,dz = 0$ for every closed triangle $T \subseteq U$.

\theorem{Primitives in a Disk} If $D \subseteq \mathbb{C}$ is an open disk and $f \in \holo{D}$ then $f$ has a primitive in $D$.

\theorem{Cauchy's Theorem in a Disk} If $D \subseteq \mathbb{C}$ is an open disk and $f \in \holo{D}$ then $\int_\gamma f(z)\,dz = 0$ for every closed curve $\gamma$ in $D$.

\remark Cauchy's theorem remains true if $f$ is continuous in $D$ and holomorphic in $D \setminus \{p\}$ for some $p \in D$.

\theorem{Cauchy's Integral Formula in a Disk} For $\disk{p}{r} \subseteq D \subseteq \mathbb{C}$, $D$ an open disk, and $f \in \holo{D}$: $$f(z) = \dfrac{1}{2\pi i}\int_{\sphere{p}{r}} \dfrac{f(\zeta)}{\zeta - z}\, d\zeta,~~z \in \disk{p}{r}.$$

\theorem{Holomorphic Implies Complex Analytic} A holomorphic function is complex analytic. If $f \in \holo{U}$ and $\disk{p}{r} \subseteq U$ then $$f(z) = \sum_{n=0}^\infty a_n (z - p)^n,~~z \in \disk{p}{r};$$ $$a_n = \dfrac{f^{(n)}(p)}{n!} = \dfrac{1}{2\pi i}\int_{\sphere{p}{s}} \dfrac{f(\zeta)}{(\zeta - p)^{n+1}}\,d\zeta,~~0 < s < r.$$

\corollary{Derivatives of All Orders} If $f \in \holo{U}$ then $f^\prime \in \holo{U}$ and so $f^{(k)} \in \holo{U}$ exists for all $k \geq 1$.

\theorem{Morera's Theorem} If $f : U \to \mathbb{C}$ is continuous and $\int_{\partial T} f(z)\,d(z) = 0$ for every closed triangle $T \subseteq U$, then $f \in \holo{U}$.

\remark If $U \subseteq \mathbb{C}$ is open, $L$ is a line, $f \in \holo{U \setminus L}$ but continuous in $U$, then $f \in \holo{U}$.

\theorem{Cauchy's Estimates} If $f \in \holo{\disk{p}{r}}$ is continuous on $\cdisk{p}{r}$, then for $n \geq 0$: $$\left|f^{(n)}(p)\right| \leq \dfrac{n!}{r^n} \sup_{|z - p| = r} |f(z)|.$$

\theorem{Ratio Bound} If $f$ is holomorphic with $f(\disk{p}{r}) \subseteq \disk{q}{R}$, then $|f^\prime(p)| \leq R/r$.

\theorem{Liouville's Theorem} Every bounded entire function is constant.

\theorem{Differentiating Under the Integral} Let $U \subseteq \mathbb{C}$ be open and $\varphi : U \times [a, b] \to \mathbb{C}$ continuous such that for $t \in [a, b]$, $z \mapsto \varphi(z, t) \in \holo{U}$ with derivative $\varphi^\prime(z, t)$. Define $f : U \to \mathbb{C}$  by $$f(z) = \int_a^b \varphi(z, t)\,dt.$$ Then $f \in \holo{U}$ and $$f^\prime(z) = \int_a^b \varphi^\prime(z, t)\,dt,~~z \in U.$$ Inductively, $$f^{(n)}(z) = \int_a^b \varphi^{(n)}(z, t)\,dt,~~z \in U.$$

\corollary{Differentiating Curve Integral} If $\gamma$ is a piecewise $C^1$ curve, $g : |\gamma| \to \mathbb{C}$ is continuous, and $f : \mathbb{C} \setminus |\gamma| \to \mathbb{C}$ is defined as $$f(z) = \int_\gamma \dfrac{g(\zeta)}{(\zeta - z)^n}\,d\zeta,~~n \geq 1$$ then $f \in \holo{\mathbb{C} \setminus |\gamma|}$ and $$f^\prime(z) = n \int_\gamma \dfrac{g(\zeta)}{(\zeta - z)^{n+1}}\,d\zeta,~~z \in \mathbb{C} \setminus |\gamma|.$$

\end{topic}


\begin{topic}{Mapping Properties of Holomorphic Functions}

\remark Let $f \in \holo{U}$ be not identically zero in $\disk{p}{r} \subseteq U$ with power series $f(z) = \sum_{n=0}^\infty a_n (z - p)^n$ there. The \term{order} of $f$ at $p$ is the index of the first non-zero $a_m$, denoted by $\order(f, p)$. $\order(f, p) \geq 1$ iff $f(p) = 0$ and $p$ is a \term{simple zero} of $f$ if $\order(f, p) = 1$. $\order(f, p)$ is the unique integer $m \geq 0$ such that $f(z) = (z - p)^m f_1(z)$ with $f_1 \in \holo{U}$ and $f_1(p) \neq 0$.

\remark $U \subseteq \mathbb{C}$ is a \term{domain} if it is open, non-empty, and connected.

\lemma{Accumulating Zero} For $f \in \holo{U}$ and $U$ a domain, if $f^{-1}(0)$ accumulates in $U$ then $f \equiv 0$ in $U$. Consequently, a non-constant holomorphic function on a domain has at most countably many zeroes, all of which are isolated.

\theorem{Identity Theorem} If $f = g$ accumulates in $U$, a domain, with $f, g \in \holo{U}$ then $f = g$ in $U$.

\lemma{Slope Continuity} If $f \in \holo{U}$ then $g : U \times U \to \mathbb{C}$ is continuous, where $g$ is: $$\renewcommand{\arraystretch}{1.2}g(\zeta, z) =
\begin{cases}
\dfrac{f(\zeta) - f(z)}{\zeta - z}&\zeta \neq z\\
f^\prime(z)&\zeta = z
\end{cases}\renewcommand{\arraystretch}{1.0}$$

\remark $f : V \to W$ is a \term{biholomorphism} if $f$ is bijective and holomorphic with holomorphic inverse.

\theorem{Holomorphic Inverse Function Theorem} Suppose $f \in \holo{U}$, $p \in U$, and $f^\prime(p) \neq 0$. Then there are open neighborhoods $V \subseteq U$ of $p$, $W \subseteq \mathbb{C}$ of $f(p)$ such that $f : V \to W$ is a biholomorphism with $$\left(f^{-1}\right)^\prime(w) = \dfrac{1}{f^\prime(f^{-1}(w))},~~w \in W.$$

\theorem{Local Normal Form} For $f \in \holo{U}$ non-constant, $U$ a domain, and $p \in U$, there exists positive $m \in \mathbb{N}$ and neighborhoods $V \subseteq U$ of $p$ and $W \subseteq \mathbb{C}$ of $q = f(p)$ and biholomorphisms $\varphi : V \to \udisk$, $\psi : W \to \udisk$ such that $\varphi(p) = \psi(q) = 0$ and $(\psi \circ f \circ \varphi^{-1})(w) = w^m$ for $w \in \udisk$.

\remark The $m$ in the Local Normal Form is the order of the zero at $p$ of $f - f(p)$ and is called the \term{local degree} of $f$ at $p$, denoted $\degree(f, p) = \order(f - f(p), p)$. $p$ is called a \term{critical point} when $\degree(f, p) > 1$ (i.e.\ $f^\prime(p) = 0$) with \term{critical value} $f(p)$.

\theorem{Open Mapping Theorem} If $f \in \holo{U}$ non-constant with $U$ a domain then $f(U)$ is open in $\mathbb{C}$.

\corollary{Critical Points \& Biholomorphisms} If $f \in \holo{U}$ is injective in a domain $U$, then $f$ has no critical points in $U$ and is biholomorphic with its image.

\theorem{Maximum Principle for Open Maps} Suppose $f : U \to \mathbb{C}$ is open. Then $|f|$ does not attain a local maximum in $U$ and, if $f$ is non-vanishing in $U$, does not attain a local minimum.

\theorem{Maximum Principle for Holomorphic Functions} Let $U \subseteq \mathbb{C}$ be a bounded domain and $f : \overline{U} \to \mathbb{C}$ be continuous with $f \in \holo{U}$. Then
\begin{enumerate}
\item[(i)] $|f(z)| \leq \sup_{\zeta \in \partial U} |f(\zeta)|$ for all $z \in U$;
\item[(ii)] $|f(z)| \geq \inf_{\zeta \in \partial U} |f(\zeta)|$ for all $z \in U$ provided $f$ does not vanish in $U$.
\end{enumerate}
If in either case equality holds at some $z \in U$ then $f$ is constant.

\end{topic}


\newpage
\section{Topological Aspects of Cauchy's Theory}

\begin{topic}{Homotopy of Curves}

\remark $\gamma : [a, b] \to X$, continuous, is a \term{curve} in a topological space $X$ with \term{initial point} $\gamma(a)$ and \term{end point} $\gamma(b)$, and it is \term{closed} if $\gamma(a) = \gamma(b)$. $|\gamma| = \{ \gamma(t) : t \in [a, b] \} \subseteq X$ is the \term{image}. A closed curve in $X$ can be identified with a continuous map of the circle into $X$.

If $\gamma$ is a curve then $\gamma^-$ is the \term{reverse} of $\gamma$, and if $\eta$ is another curve starting where $\gamma$ ends, then $\gamma \curveproduct \eta$ is the \term{product} of $\gamma$ and $\eta$ and is formed by traversing $\gamma$ and then $\eta$. $\varepsilon_p$ is the \term{constant curve} at $p$.

\remark A \term{homotopy} between $\gamma_0, \gamma_1$, curves in $X$ from $p$ to $q$ is a continuous map $H : [0, 1] \times [0, 1] \to X$ with:
\begin{itemize}
\item $H(t, 0) = \gamma_0(t)$ and $H(t, 1) = \gamma_1(t)$, $t \in [0, 1]$.
\item $H(0, s) = p$ and $H(1, s) = q$, $s \in [0, 1]$.
\end{itemize}
When such a function exists, $\gamma_0$ and $\gamma_1$ are called \term{homotopic} in X, written $\gamma_0 \simeq \gamma_1$. $\gamma_0 \simeq \gamma_1$ if $\gamma_0$ can be continuously deformed {\em in $X$} to $\gamma_1$.

\lemma{Homotopy of Compositions} $\gamma$ a curve in $X$, $\varphi : [0, 1] \to [0, 1]$ continuous.
\begin{enumerate}
\item[(i)] If $\varphi$ maps $0$ and $1$ to themselves then $\gamma \circ \varphi \simeq \gamma$.
\item[(ii)] If $\varphi(0) = \varphi(1)$ then $\gamma \circ \varphi \simeq \varepsilon_p$, $p = (\gamma \circ \varphi)(0)$.
\end{enumerate}

\theorem{Homotopy is an Equivalence} Homotopy $\simeq$ is an equivalence relation on the set of curves with a given initial point and end point. Moreover:
\begin{enumerate}
\item[(i)] $\gamma \simeq \eta \iff \gamma^- \simeq \eta^-$.
\item[(ii)] If $\gamma(1) = \eta(0)$ and $\gamma \simeq \hat\gamma, \eta \simeq \hat\eta$, then $\gamma \curveproduct \eta \simeq \hat\gamma \curveproduct \hat\eta$.
\item[(iii)] If $p = \gamma(0)$ and $q = \gamma(1)$ then $\varepsilon_p \curveproduct \gamma \simeq \gamma \curveproduct \varepsilon_q \simeq \gamma$ and $\gamma \curveproduct \gamma^- \simeq \varepsilon_p$.
\item[(iv)] If $\gamma(1) = \eta(0)$ and $\eta(1) = \xi(0)$ then $(\gamma \curveproduct \eta) \curveproduct \xi \simeq \gamma \curveproduct (\eta \curveproduct \xi)$.
\end{enumerate}

\remark The equivalence class of a curve $\gamma$ under $\simeq$ is called its \term{homotopy class} and is denoted $[\gamma]$. When $\gamma(1) = \eta(0)$ we have $[\gamma] \curveproduct [\eta] = [\gamma \curveproduct \eta]$ is well defined. For $p \in X$, the set of homotopy classes of closed curves starting at $p$ forms a group, called the \term{fundamental group} of $X$ with \term{base point} $p$, under the aforementioned product and is denoted by $\pi_1(X, p)$.

\remark $f : X \to Y$ continuous. Then $f_\ast : \pi_1(X, p) \to \pi_1(Y, f(p))$ given by $f_\ast([\gamma]) = [f \circ \gamma]$ is a group homomorphism. If $g : Y \to Z$ is continuous, then $(g \circ f)_\ast = g_\ast \circ f_\ast$. If $f$ is a homeomorphism then $f_\ast$ is an isomorphism with inverse $(f^{-1})_\ast$--homeomorphic spaces have isomorphic fundamental groups.

\remark $X$, path-connected, is \term{simply connected} if $\pi_1(X, p)$ is trivial for some (and hence every) $p \in X$. If every closed curve $\gamma$ from $p \in X$ is \term{null-homotopic}, i.e.\ $\gamma \simeq \varepsilon_p$, then $X$ is simply connected.

\theorem{Homotopy of Curves in Simply Connected Spaces} A path-connected space $X$ is simply connected if and only if any two curves with the same initial and end points are homotopic.

\remark $\gamma_0, \gamma_1$, closed curves in $X$, are \term{freely homotopic} in $X$ if there exists a continuous map $H : [0, 1] \times [0, 1] \to X$ with:
\begin{itemize}
\item $H(t, 0) = \gamma_0(t)$ and $H(t, 1) = \gamma_1$ for all $t \in [0, 1]$.
\item $H(0, s) = H(1, s)$ for all $s \in [0, 1]$.
\end{itemize}
$\gamma_0$ is called \term{freely null-homotopic} in $X$ if $\gamma_1 = \varepsilon_p, p \in X$. Free homotopy is an equivalence relation.

\end{topic}


\begin{topic}{Covering properties of the exponential map}

\remark $\exp(z) = \exp(\zeta) \iff z - \zeta = 2 k \pi i, k \in \mathbb{Z}$ and $\exp(\mathbb{C}) = \mathbb{C}^\ast = \mathbb{C} \setminus \{ 0 \}$.

\remark For $w_0 \in \mathbb{C}^\ast$, $\exp^{-1}(w_0) = z_0 + 2 \pi i \mathbb{Z}$ where $z_0$ is {\em some} preimage of $w_0$.

\remark Let $w_0 \in \mathbb{C}^\ast$ and select $z_0 \in \exp^{-1}(w_0)$, and set $z_k = z_0 + 2 \pi i k$. Then $w_0$ has a neighborhood $W$ that is \term{evenly covered} by $\exp$, that is $\exp^{-1}(W)$ is the disjoint union of open sets $O_k = O_0 + 2 \pi i k$ where $z_k \in O_k$ and $\exp : O_k \to W$ is a homeomorphism, yielding a \term{local inverse} of $\exp$, $L_k : W \to O_k$.

\remark A \term{lift} (under $\exp$) of a continuous map $f : X \to \mathbb{C}^\ast$ is a continuous map $\tilde{f} : X \to \mathbb{C}$ satisfying $\exp(\tilde{f}) = f$. $\tilde{f}$ is also called a \term{branch of the logarithm of $f$}.

\theorem{Uniqueness of Lifts} Any two lifts of a continuous map from a connected space $X$ to $\mathbb{C}^\ast$ differ by an integer multiple of $2 \pi i$.

\theorem{Existence of Lifts from Unit Interval \& Square}
\begin{enumerate}
\item[(i)] Given a curve $\gamma : [0, 1] \to \mathbb{C}^\ast$ and $p \in \exp^{-1}(\gamma(0))$, there is a unique lift $\tilde{\gamma}$ of $\gamma$ with $\tilde{\gamma}(0) = p$.
\item[(ii)] Given a continuous map $H : [0, 1] \times [0, 1] \to \mathbb{C}^\ast$ and $p \in \exp^{-1}(H(0, 0))$, there is a unique lift $\tilde{H}$ of $H$ with $\tilde{H}(0, 0) = p$.
\end{enumerate}
Part (i) is the \term{curve lifting property} of $\exp$.

\lemma{Lebesgue's Cover Lemma} For an open cover $\{U_\alpha\}$, there exists a \term{Lebesgue number} $\delta > 0$ such that any set $E$ with $\diam E < \delta$ lies in some $U_\alpha$.

\corollary{Lift End Points Corollary} Let $\gamma_0 \simeq \gamma_1$ with $p \in \mathbb{C}$ such that $\exp(p) = \gamma_i(0)$. Let $\tilde{\gamma_i}$ be the unique lift of $\gamma_i$ with $\tilde{\gamma_i}(0) = p$. Then $\tilde{\gamma_0}(1) = \tilde{\gamma_1}(1)$ and $\gamma_0$ is a null-homotopic closed curve if and only if $\tilde{\gamma}$ is a closed curve.

\remark A space is \term{locally path-connected} if every neighborhood of every point contains a path-connected neighborhood of that point.

\theorem{Existence of Lifts} Suppose $X$ is simply connected and locally path-connected and $f : X \to \mathbb{C}^\ast$ continuous. Then $f = \exp(g)$ for some continuous $g : X \to \mathbb{C}$ which unique up to addition of an integer multiple of $2 \pi i$.

\corollary{Existence of Holomorphic Logarithms and n-th Roots} Suppose $U \subseteq \mathbb{C}$ is a simply connected domain on which $f \in \holo{C}$ does not vanish. Then:
\begin{enumerate}
\item[(i)] There is $g \in \holo{U}$, unique up to addition of an integer multiple of $2 \pi i$, with $f = \exp(g)$.
\item[(ii)] For $n$ a positive integer, there is $h \in \holo{U}$, unique up to multiplication by an $n$-th root of unity, with $f = h^n$.
\end{enumerate}

\end{topic}


\begin{topic}{The winding number}

\remark All the lifting results of the previous section hold if we replace $\mathbb{C}^\ast$ and $z \mapsto \exp(z)$ by $\mathbb{C} \setminus \{p\}$ and $z \mapsto \exp(z) + p$, respectively.

\remark $\gamma$ a curve with $p \not\in |\gamma|$ and $\tilde{\gamma}$ a lift of $\gamma$ under $z \mapsto \exp(z) + p$. The quantity $$W(\gamma, p) = \dfrac{1}{2 \pi i}\left(\tilde{\gamma}(1) - \tilde{\gamma}(0)\right)$$ is the \term{winding number} of $\gamma$ with respect to $p$. $W(\gamma, p) \in \mathbb{Z}$ if and only if $\gamma$ is closed.

\theorem{Properties of the Winding Number} Let $\gamma$ be a curve with $p \not\in |\gamma|$, then:
\begin{enumerate}
\item[(i)] $W(\gamma, p) = W(\gamma + w, p + w)$ for all $w \in \mathbb{C}$.
\item[(ii)] $W(\gamma^-, p) = -W(\gamma, p)$.
\item[(iii)] If $\eta$ is a curve with $p \not\in \eta, \eta(0) = \gamma(1)$ then $W(\gamma \curveproduct \eta, p) = W(\gamma, p) + W(\eta, p)$.
\item[(iv)] If $\eta \simeq \gamma$ (or freely homotopic closed curves) in $\mathbb{C} \setminus \{ p \}$ then $W(\gamma, p) = W(\eta, p)$.
\end{enumerate}
If $\gamma$ is closed, then
\begin{enumerate}
\item[(v)] $W(\gamma, p) = 0$ if and only if $\gamma$ is freely homotopic to a constant curve in $\mathbb{C} \setminus \{ p \}$.
\item[(vi)] $W(\gamma, z)$ is constant on the components of $\mathbb{C} \setminus |\gamma|$ and $0$ on the unbounded region.
\end{enumerate}

\remark The \term{jump principle}: each time $|\gamma|$ is crossed from right to left, the winding number increments.

\remark A closed curve is a \term{Jordan curve} (or \term{simple closed curve}) if it is injective on $[0, 1)$. If $\gamma$ is a Jordan curve, then the \term{Jordan Curve Theorem} says $\mathbb{C} \setminus |\gamma|$ has two components: a bounded \term{interior} and unbounded \term{exterior}, denoted $\interior(\gamma)$ and $\exterior(\gamma)$, respectively. $|\gamma| = \partial \interior(\gamma) = \partial \exterior(\gamma)$.

\theorem{Winding Number for Jordan Curves} If $\gamma$ is a Jordan curve, then $W(\gamma, \cdot)$ vanishes on $\exterior(\gamma)$ and is $\pm 1$ on $\interior(\gamma)$. $\gamma$ is \term{positively} or \term{negatively oriented} according to the sign of $W(\gamma, \cdot)$ inside $\interior(\gamma)$.

\theorem{Analytic Description of the Winding Number} Let $\gamma$ be a piecewise $C^1$ curve, $p \not\in |\gamma|$. Then $$W(\gamma, p) = \dfrac{1}{2 \pi i} \int_{\gamma} \dfrac{dz}{z - p}.$$

\end{topic}


\begin{topic}{Cycles and homology}

\remark The \term{free abelian group generated by $S$} is the collection $\mathcal{G}(S)$ of functions $\varphi : S \to \mathbb{Z}$ with $\varphi$ non-zero finitely often. Let $\varphi_x$ be the characteristic function on $\{x\}$. Then for $\varphi \in \mathcal{G}(S)$ vanishing outside $\{x_1, \ldots, x_m\}$ with $n_k = \varphi(x_k)$ we have $\varphi = n_1 \varphi_{x_1} + \cdots n_m \varphi_{x_m}$. Such representations are unique, and identifying $\varphi_x$ with $x$ we get $\varphi = n_1 x_1 + \cdots + n_m x_m$.

\remark Let $U \subseteq \mathbb{C}$ be non-empty and open. A \term{chain} in $U$ is an element of the free abelian group generated by curves in $U$, e.g.\ a formal sum $$\gamma = n_1 \gamma_1 + \cdots + n_m \gamma_m$$ with $n_k \in \mathbb{Z}$ and $\gamma_k$ a curve in $U$. $n_k$ is the \term{multiplicity} of $\gamma_k$ and $$|\gamma| = |\gamma_1| \cup \cdots \cup |\gamma_m|$$ is the \term{image} of $\gamma$.

\remark If $U_\gamma$ is the set of curves in $U$, then the \term{boundary map} $\partial : \mathcal{G}(U_\gamma) \to \mathcal{G}(U)$ is the group homomorphism:
$$\sum_{k=1}^m n_k \gamma_k \mapsto \sum_{k=1}^m n_k \gamma_k(1) - \sum_{k=1}^m n_k \gamma_k(0).$$

\remark A chain is a \term{cycle} if every $p \in U$ is an initial point and end point an equal number of times, counting multiplicity; e.g. $$\sum_{\{k:\gamma_k(0)=p\}} n_k = \sum_{\{k:\gamma_k(1)=p\}} n_k.$$ A chain $\gamma$ is a cycle if and only if $\partial\gamma = 0$. Since $\partial$ is a group homomorphism, the cycles of $U$ are the kernel of the boundary map of $U$ and are thus an additive group.

\theorem{Cycles decompose into closed curves} If $\gamma = \sum_{k=1}^m n_k \gamma_k$ is a cycle, then collecting $|n_k|$ copies of either $\gamma_k$ or, when $n_k < 0$, $\gamma_k^-$, the curves can be partitioned into subcollections each of whose product can be arranged to be a closed curve.

\remark The \term{winding number} extends to chains. If $\gamma = \sum_{k=1}^m n_k \gamma_k$ and $p \not\in |\gamma|$ then $$W(\gamma, p) = \sum_{k=1}^m n_k W(\gamma_k, p).$$

\theorem{Winding numbers of chains and cycles} \begin{enumerate}
\item[(i)] For any chain $\gamma$, $W(-\gamma, p) = -W(\gamma, p)$.
\item[(ii)] For any two chains $\gamma$ and $\eta$, $W(\gamma + \eta, p) = W(\gamma, p) + W(\eta, p)$.
\item[(iii)] If $\gamma$ is a cycle then $W(\gamma, \cdot)$'s range is $\mathbb{Z}$, is constant on the connected components of its domain, vanishing on the unbounded component.
\end{enumerate}

\remark Let $U \subseteq \mathbb{C}$ be non-empty and open. A cycle $\gamma$ is \term{null-homologous}, $\gamma \sim 0$, if $W(\gamma, \cdot) \equiv 0$ on $\mathbb{C} \setminus U$.

\remark Two cycles $\gamma, \eta$ in $U$ are \term{homologous} if $\gamma - \eta \sim 0$, i.e.\ $W(\gamma, \cdot) = W(\eta, \cdot)$ on $\mathbb{C} \setminus U$.

\remark $\sim$ is an equivalence relation. The equivalency class of a cycle $\gamma$ under $\sim$, denoted by $\langle \gamma \rangle$, is called the \term{homology class} of $\gamma$. $\langle \gamma \rangle + \langle \eta \rangle = \langle \gamma + \eta \rangle$ is well-defined. The set of homology classes in $U$ is an abelian group called the \term{first homology group} of $U$, denoted by $H_1(U)$.

\remark $H_1(U)$ is the quotient group of all cycles in $U$ by the all the null-homologous cycles in $U$.

\theorem{Homotopic implies homologous} \begin{enumerate}
\item[(i)] If $\gamma \simeq \eta$ in $U$ then $\gamma \sim \eta$ in $U$.
\item[(ii)] If $\gamma$ and $\eta$ are freely homotopic in $U$ then $\gamma \sim \eta$ in $U$.
\end{enumerate}

\remark This implies that $\varphi : \pi_1(U, p) \to H_1(U)$ defined by sending $\gamma$'s homotopy class to its homology class is well-defined.

\lemma{Homology class of a closed product} Suppose $\gamma_1 \curveproduct \gamma_2 \curveproduct \cdots \curveproduct \gamma_n$ is a closed curve. Then, in $U$, $$\gamma_1 \curveproduct \gamma_2 \curveproduct \cdots \curveproduct \gamma_n ~ \gamma_1 + \gamma_2 + \cdots + \gamma_n.$$

\remark This implies that $\varphi$ in the above remark is a group homomorphism. Since $H_1(U)$ is abelian, $\varphi$ must vanish on the commutator $C$ of $\pi_1(U, p)$, inducing a homomorphism $\Phi : \pi_1(U, p)/C \to H_1(U)$. Poincar\'{e} and Hurewicz showed $\Phi$ is an isomorphism.

\corollary{Homology group of simply connected domains} Every cycle in a simply connected domain $U$ is null homologous, and so $H_1(U) = 0$ 

\end{topic}


\begin{topic}{The homology version of Cauchy's theorem}

\remark Let $\gamma = \sum_{k=1}^m n_k \gamma_k$ be a chain, each $\gamma_k$ piecewise $C^1$, and let $f : |\gamma| \to \mathbb{C}$ be continuous. Then define $$\int_\gamma f(z)\,dz = \sum_{k=1}^m n_k \int_{\gamma_k} f(z)\,dz.$$

\remark Observe the special case of $f(z) = 1/(z-p)$, $p \not\in |\gamma|$ gives $$\dfrac{1}{2 \pi i} \int_\gamma \dfrac{dz}{z - p} = \sum_{k=1}^m n_k W(\gamma_k, p) = W(\gamma, p).$$

\theorem{Homology version of Cauchy's theorem} Let $U \subseteq \mathbb{C}$ be open and $\gamma$ a piecewise $C^1$ null-homologous cycle in $U$. For $f \in \holo{U}$: $$\int_\gamma f(z)\,dz = 0,$$ $$f(z) \cdot W(\gamma, z) = \dfrac{1}{2 \pi i} \int_\gamma \dfrac{f(\zeta)}{\zeta - z}\,d\zeta,~~z \in U \setminus |\gamma|.$$

\corollary{Cauchy's theorem in simple connected domains} The above equations hold for all piecewise $C^1$ cycles in a simply connected domain.

\corollary{Integration independent of homotopic curves / homologous cycles} If $\gamma \simeq \eta$ or $\gamma \sim \eta$ (and are piecewise $C^1$) then $$\int_\gamma f(z)\,dz = \int_\eta f(z)\,dz.$$

\theorem{Primitives in simply connected domains} Every holomorphic function in a simply connected domain has a primitive.

\corollary{Cauchy's integral formula for higher derivatives} For $f \in \holo{U}$, $\gamma \sim 0$ in $U$, and $n \geq 1$: $$f^{(n)}(z) \cdot W(\gamma, z) = \dfrac{n!}{2 \pi i} \int_\gamma \dfrac{f(\zeta)}{(\zeta - z)^{n+1}}\,d\zeta~~z \in U \setminus |\gamma|.$$

\remark If $\gamma$ is not piecewise $C^1$ or rectifiable, there is a piecewise $C^1$ curve, $\eta$ it is homotopic to. If $f \in \holo{U}$ then define $\int_\gamma f(z)\,dz$ to be $\int_\eta f(z)\,dz$ which by the above statements is independent of the choice of $\eta$. Extending to cycles by linearity we can always assume paths of integration are piecewise $C^1$.

\end{topic}


\newpage
\section{Meromorphic Functions}

\begin{topic}{Isolated singularities}

\remark $p$ is an \term{isolated singularity} of $f$ if there is an open neighborhood $U$ of $p$ such that $f \in \holo{U \setminus \{p\}}$. $p$ is a \term{removable singularity} if there exists $g \in \holo{U}$ such that $g(z) = f(z)$ for $z \in U \setminus \{p\}$.

\theorem{Riemann's removable singularity theorem} If a holomorphic function is bounded in a punctured neighborhood of a isolated singularity, then the singularity is removable.

\remark Boundedness can be replaced by $\lim_{z \to p}f(z) = 0$ in the above.

\theorem{Classification of isolated singularities} An isolated singularity $p$ of $f$, holomorphic, is one of: \begin{itemize}
\item[(i)] Removable for which $\lim_{z \to p} f(z)$ exists.
\item[(ii)] A \term{pole} for which $\lim_{z \to p} f(z) = \infty$. In this case there is a positive integer $m$ such that $f_1(z) = (z - p)^m f(z)$ has a removable singularity at $p$ and $f_1(p) \neq 0$.
\item[(iii)] An \term{essential singularity} for which $\lim_{z \to p}$ does not exist. In this case, for every small $r > 0$, $f(\pdisk{p}{r})$ is dense in $\mathbb{C}$.
\end{itemize}

\remark The density statement in (iii) is the Casorati-Weierstrass theorem. Picard showed that at most one value of $\mathbb{C}$ is missed.

\remark If $f$ has a pole at $p$, then the $m$ in the classification theorem is the \term{order} of $f$ at $p$, denoted $\order(f, p)$.

\remark If $\order(f, p) = 1$ then $p$ is a \term{simple pole}.

\remark If $m = \order(f, p)$ then $$f(z) = \sum_{n=0}^\infty b_n (z - p)^{n - m}$$ in a punctured neighborhood about $p$. The \term{principle part} of $f$ at $p$ is the rational function $$\dfrac{b_0}{(z - p)^m} + \cdots + \dfrac{b_{m-1}}{z-p}.$$

\remark The principle part of $f$ at $p$ is the unique polynomial $P$ in $(z-p)^{-1}$ with $P(0) = 0$ such that $f(z) - P((z-p)^{-1})$ has a removable singularity at $p$.

\remark The principle part of $f$ at $p$ is the unique rational function $R$ with a single pole, $p$, such that $\lim_{z \to \infty} R(z) = 0$ and $f(z) - R(z)$ has a removable singularity at $p$.

\remark If $f$ has order $m$ at $p$ then $f$ is locally $m$-to-$1$ near $p$ (consider $1/f$'s zeroes). The the \term{degree} of $f$ at a pole $p$ is equal to $\order(f, p)$, denoted $\degree(f, p)$.

\end{topic}


\begin{topic}{The Riemann Sphere}

\remark $f$ is \term{meromorphic} in the open set $U \subseteq \mathbb{C}$, written $f \in \mero{U}$, if there exists $E \subset U$ without any accumulation points in $U$ such that $f \in \holo{U \setminus E}$ and $E$ consists of poles of $f$.

\remark $\holo{U} \subset \mero{U}$.

\remark $f, g \in \holo{U}, g \not\equiv 0$ then $f / g \in \mero{U}$. Converse given later: $h \in \mero{U} \implies h = f / g$, $f, g \in \holo{U}$.

\remark If $U$ is a domain then $\mero{U}$ is a field under pointwise arithmetic. $\mero{U}$ is the quotient field of $\holo{U}$.

\remark $f \in \mero{U}$ implies $f^\prime \in \mero{U}$.

\remark The \term{Riemann sphere} $\hat{\mathbb{C}} = \mathbb{C} \cup \infty$ is the one-point compactification of $\mathbb{C}$; it's topology is generated by open disks in $\mathbb{C}$ along with sets of the form $\{z \in \mathbb{C} : |z| > r\} \cup \{\infty\}$.

\remark The \term{stereographic projection} is an explicit homeomorphism between the two dimensional sphere $\mathbb{S}^2 = \{ {\bf x} \in \mathbb{R}^3 : \|x\| = 1 \}$ and $\hat{\mathbb{C}}$. The projection is given by intersecting $\mathbb{S}^2$'s equator with the plane and sending the north pole $(0, 0, 1)$ to $\infty$ and any other point $p \in \mathbb{S}^2$ to the point $q \in \mathbb{C}$ that is colinear with $p$ and the north pole. Explicitly:$$\varphi(x_1, x_2, x_3) = \begin{cases}
\dfrac{x_1 + i x_2}{1 - x_3}&x_3 \neq 1\\[10pt]
\infty&x_3 = 1,
\end{cases}$$
$$\varphi^{-1}(z) = \begin{cases}
\left(\dfrac{z + \overline{z}}{|z|^2 + 1}, \dfrac{1}{i} \dfrac{z - \overline{z}}{|z|^2 + 1}, \dfrac{|z|^2 - 1}{|z|^2 + 1}\right)&z \neq \infty\\[10pt]
(0, 0, 1)&z = \infty
\end{cases}.$$

\remark Several maps from $\mathbb{S}^2$ to itself correspond to interesting maps in $\hat{\mathbb{C}}$ under the stereographic projection: \begin{itemize}
\item The $180^\circ$ rotation $(x_1, x_2, x_3) \mapsto (-x_1, -x_2, x_3)$ rotates $\hat{\mathbb{C}}$ by fixing $0$ and $\infty$ and swaps $\pm 1$.
\item The $180^\circ$ rotation $(x_1, x_2, x_3) \mapsto (x_1, -x_2, -x_3)$ yields $1/z$, which fixes $\pm 1$ and swaps $0$ and $\infty$.
\item The reflection $(x_1, x_2, x_3) \mapsto (x_1, -x_2, x_3)$ gives conjugation, $z \mapsto \overline{z}$.
\end{itemize}

\remark Let $f$ be a continuous map of (some open subset of) $\hat{\mathbb{C}}$ to $\hat{\mathbb{C}}$. \term{holomorphic} in $\hat{\mathbb{C}}$ requires us to consider: \begin{enumerate}
\item $f(p) = \infty$ for $p \neq \infty$: $f$ is holomorphic at $p$ if $$g(z) = \begin{cases}
1/f(z)&z \neq p\\[10pt]
0&z = p
\end{cases}$$ is holomorphic near $p$.

\item $f(\infty) \neq \infty$: $f$ is holomorphic at $\infty$ if $$g(z) = \begin{cases}
f(1/z)&z \neq 0\\[10pt]
0&z = 0
\end{cases}$$ is holomorphic near $0$.

\item $f(\infty) = \infty$: $f$ is holomorphic at $\infty$ if $$g(z) = \begin{cases}
1/f(1/z)&z \neq 0\\[10pt]
0&z = 0
\end{cases}$$ is holomorphic near $0$.

\item Otherwise examine $f$ in the usual sense in $\mathbb{C}$.
\end{enumerate}

\end{topic}

\end{document}
